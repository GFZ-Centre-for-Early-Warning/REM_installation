\documentclass{article}

\usepackage{pdftexcmds}
\usepackage{minted}
\usepackage{hyperref}

\setlength\parindent{0pt}

\begin{document}

\begin{titlepage}
\textbf{\huge REM – Installation instructions}\\
\centering Michael Haas - mhaas@gfz-potsdam.de
\end{titlepage}


\section{General information}

This installation guide is for an Ubuntu 16.04 operating system.
The instructions assume a fresh install of Ubuntu 16.04 and go through 
all steps necessary to install all software used in the training 
course. There are two parts the first part (Part A) 
describes the installation for all components used for data preparation
and analysis. The second part (Part B) is an instruction on how
to set up a server for the RRVS analysis.
Although the target system is Ubuntu 16.04, all steps 
described here should be similar on other Linux distributions.
Be aware that some of the dependencies might not be packed the 
same way as in Ubuntu and might be installed additionally.

Ubuntu 16.04 and all components 
(except software related to the Ladybug 3 camera) 
are open source and can be used free of charge. 
All software developed by the GFZ-Center for Early Warning Systems 
can be found on the online
repository service github at:

\url{https://github.com/GFZ-Centre-for-Early-Warning} 

\textit{Most components are also available for Windows environments.
A good starting point 
for installing most tools discussed here on Windows is OSGeo4W.
\url{https://trac.osgeo.org/osgeo4w/}}


\section{PART A. Software used for Pre- and Postprocessing}

\subsection{Before the installation}

We assume a basic experience in linux environments, i.e., familiarity
with a terminal and how to use a package manager furthermore root
privileges are required.
Make sure your repository list and all installed packages 
are up-to-date, run in a terminal:
\begin{minted}{bash}
sudo apt-get update
sudo apt-get upgrade
\end{minted}

\subsection{Tools}
The code developed by GFZ is hosted on github, 
thus the first requirement is git and we will install 
most python packages using pip.
Install them from the shell using apt-get and upgrade pip 
using pip:

\begin{minted}{bash}
sudo apt-get install git
sudo apt-get install python-pip
sudo pip install --upgrade pip
\end{minted}

\textit{NOTE: We currently support only Python 2.7}

\subsection{Satellite Analysis}

For the satellite analysis we will require the OrfeoToolbox (OTB) 
we get the binary from the OTB webpage and install it to the home
directory:
\begin{minted}{bash}
wget https://www.orfeo-toolbox.org/packages/OTB-5.10.1-Linux64.run
chmod +x OTB-5.10.1-Linux64.run
./OTB-5.10.1-Linux64.run
\end{minted}

%which we will install from source (since Ubuntu/Debian does not
%support libsvm):
%\begin{minted}{bash}
%sudo apt-get install cmake
%sudo apt-get install swig
%git clone https://git@git.orfeo-toolbox.org/git/otb.git
%cd otb
%git checkout release-5.10
%mkdir build
%cd build
%gedit ../SuperBuild/CMakeLists.txt
%\end{minted}
%
%In the file add at the top add 
%\begin{minted}{bash}
%set (CMAKE_CXX_STANDARD 11)
%\end{minted}
%Save and close the file. Back in the terminal run from the build
%directory:
%\begin{minted}{bash}
%cmake -DCMAKE_INSTALL_PREFIX=/usr -DBUILD_EXAMPLES:BOOL=OFF -DOTB_USE_LIBSVM:BOOL=ON -DOTB_WRAP_PYTHON:BOOL=ON -DOTB_USE_QT4:BOOL=OFF ../SuperBuild
%sudo make -j$(nproc)
%\end{minted}
This should run without errors and install all dependencies and
OTB with the necessary python wrappers.

For python to find the wrappers and 
the wrappers to find the functions we need to add an 
environment variable to our profile:
\begin{minted}{bash}
gedit ~/.profile
\end{minted}
add at the end of the file:
\begin{minted}{bash}
export ITK_AUTOLOAD_PATH=$HOME/OTB-5.10.1-Linux64/lib/otb/applications
export PYTHONPATH=$HOME/OTB-5.10.1-Linux64/lib/python
\end{minted}
and save it. Source the file in the terminal:
\begin{minted}{bash}
source ~/.profile
\end{minted}
This should load the newly set environmental variables. In future
they should be set when you log in to the system.
Try to import OTB in python you maybe have to install numpy:
\begin{minted}{bash}
sudo pip install numpy
sudo apt-get install ipython
ipython
\end{minted}
From within ipython:
\begin{minted}{python}
import otbApplication
otbApplication.Registry.GetAvailableApplications()
\end{minted}
This should print a list of available functions of OTB. 

\subsection{GIS}

Since most of the information we use during the training course
is spatial information our primary visualisation and 
data exploration tool is a Geographic Information System (GIS).
We use the free and open source QGIS. A recent version
can be installed adding the QGIS repository and installing the
package from there:
\begin{minted}{bash}
sudo sh -c 'echo "deb http://qgis.org/debian xenial main"\
            >> /etc/apt/sources.list'  
sudo sh -c 'echo "deb-src http://qgis.org/debian xenial main "\
	    >> /etc/apt/sources.list'  
wget -O - http://qgis.org/downloads/qgis-2016.gpg.key | gpg --import
gpg --fingerprint 073D307A618E5811
gpg --export --armor 073D307A618E5811 | sudo apt-key add -
sudo apt-get update && sudo apt-get install qgis python-qgis
\end{minted}

QGIS functionality can be extended by installing plugins. We
will install a couple of plugins for the purpose of this
workshop. Start qgis either from the dashboard (hit windows key and
type "qgis") or by typing "qgis" in a terminal (better you will see
in case there is a problem starting qgis). 

When qgis has started move the mousepointer to the top of the window
a couple of options should appear above the symbols, select:
\begin{minted}{bash}
Plugins -> Manage and Install Plugins
\end{minted}
A new window will open type "OSM" in the "Search" field in the top 
of the window. It will show a list of plugins with OSM in the name.
Select the "OSMDownloader". And click on Install plugin 
in the lower right corner of the window. 
This gives you a handy tool to download
OSM data from within QGIS. Install the same way
also the plugin "Rectangles ovals digitizing" a nice tool to digitize
regular shapes, e.g., for defining rectangular regions of interest.

For the satellite analysis we will use a plugin developed at GFZ
called REM\_SatEx it is available from our github repository.
Open a terminal, clone the plugin, install python sphinx and
pyqt4 developer tools and deploy the plugin:
\begin{minted}{bash}
cd ~
git clone https://github.com/GFZ-Centre-for-Early-Warning/REM_satex_plugin
sudo apt-get install pyqt4-dev-tools
sudo pip install sphinx
cd REM_satex_plugin
make compile
make deploy
\end{minted}

The core dependency of the SatEx plugin is the previously installed 
OTB.
Open qgis and make sure the OTB installation worked 
and includes the necessary python wrappers. From
the top menu in QGIS select
\begin{minted}{bash}
Plugins -> Python Console
\end{minted}
and type:
\begin{minted}{python}
import otbApplication
otbApplication.Registry.GetAvailableApplications()
\end{minted}
This should print a list of available functions of OTB. 

\textit{If this returns an error:}

If the import fails make sure the pythonpath is set correctly:
\begin{minted}{python}
import os
os.environ['PYTHONPATH']
\end{minted}
this should contain the "PYTHONPATH" directory previously 
specified in ".profile".
In case the list of functions is empty qgis cannot find the
"ITK\_AUTOLOAD\_PATH" in the same file. Also try if
it works from the python interpreter in a terminal. 
In the worst case there is a python script called
"run\_as\_script.py" which allows to run the classification
using the parameters specified in the "config.ini" in the repository.
 
Finally activate the plugin by opening again the "Manage and Install Plugins" from the "Plugins" dropdown menu and typing "GFZ". Tick the
checkbox in front of "GFZ SatEx" to activate it. Among the symbols
at the top of the QGIS Interface you should see two new buttons
with "SatEx" written on them.

The SatEx repository contains a demo dataset copy it to the Desktop
and try to apply the plugin using the input data provided
in the directory to make sure the plugin works as intended. If
you click on the help button within one of
the dialogs you will find instructions on how to use
the plugin.

\subsection{Routing}
For the routing we use PGRouting it builds upon a PostgreSQL and 
PostGIS installation.
 At the time
of writing this document the most recent packages available in
the official Ubuntu16.04 repository is PostgreSQL 9.5 and PostGIS 2.2.
\textbf{Be aware that upgrading PostgreSQL or PostGIS can make the 
datacluster of a previous installation unusable! Before you
upgrade PostgreSQL or PostGIS (be aware of apt-get upgrade) create a backup of the database cluster (see section \ref{Database maintenance}).}

\begin{minted}{bash}
sudo apt-get install postgresql-9.5-postgis-2.2 postgresql-server-dev-9.5
\end{minted}

Set up a password for the default PostgreSQL user "postgres" using psql
, the terminal interpreter tool for PostgreSQL
\begin{minted}{bash}
sudo -u postgres psql postgres
\end{minted}
The psql version should be printed in the shell and the 
prompt should change to "postgres=\#", indicating that you are connected
to the default database "postgres".  

Execute the following command to set a password for the postgres user
and leave psql afterwards:

\begin{minted}{SQL}
\password
\q
\end{minted}

Open as root the postgres configuration file:
\begin{minted}{bash}
sudo gedit /etc/postgresql/9.5/main/pg\_hba.conf
\end{minted}
Change "peer" to "md5" in the line (around 85):
\begin{minted}{bash}
old:local   all             postgres                                peer
new:local   all             postgres                                md5
\end{minted}
and restart the postgres server:
\begin{minted}{bash}
sudo /etc/init.d/postgresql restart
\end{minted}

Verify where your PostgreSQL datacluster resides on your harddrive
using psql:
\begin{minted}{bash}
psql --u postgres
\end{minted}
execute the following command within psql:
\begin{minted}{SQL}
SHOW data_directory;
\q
\end{minted}
to get the location of the datacluster. \textbf{Memorize or write down the location in case the server crashes at some point in the future 
and you have to restart it manually.}
On Ubuntu the default should be "/var/lib/postgresql/9.5/main".

Alongside the command line tool psql there is also a tool
with a graphical user interface which can be handy for
quick checks or a more comfortable look at data.
It is called "pgadmin3" and can be installed using:
\begin{minted}{bash}
sudo apt-get install pgadmin3
\end{minted}


Now to install PGRouting add a repository to aptituted:
\begin{minted}{bash}
sudo sh -c 'echo "deb http://apt.postgresql.org/pub/repos/apt/ $(lsb_release -cs)-pgdg main" > /etc/apt/sources.list.d/pgdg.list'
sudo apt install wget ca-certificates
wget --quiet -O - https://www.postgresql.org/media/keys/ACCC4CF8.asc | sudo apt-key add -
sudo apt update
sudo apt-get install libboost-graph-dev
sudo apt-get install libcgal*
sudo apt-get install postgresql-9.5-pgrouting
\end{minted}

For the database you want to use for routing, remember to activate
the PostGIS and PGRouting Extensions, i.e. within psql:
\begin{minted}{bash}
psql --u postgres
\end{minted}
create a new database e.g. "routing", 
connect to it and create the extensions:
\begin{minted}{SQL}
CREATE DATABASE routing;
\c routing
CREATE EXTENSION postgis;
CREATE EXTENSION pgrouting;
\q
\end{minted}
This gives you the functions you need for routing on the database.

\textit{Hint: Routing is based on directed graphs. For this
to work properly the topology of the network has to be clean.
The GRASS GIS functions (e.g. v.clean) are very helpful in cleaning
streetnetworks used for routing. They can be installed alongside
QGIS and called from within QGIS.}

\subsection{Statistical Analysis}
One of the best free tools for statistical analysis 
is the R project.
We can install it using:
\begin{minted}{bash}
sudo sh -c 'echo "deb http://cran.rstudio.com/bin/linux/ubuntu xenial/" >> /etc/apt/sources.list'
sudo apt-key adv --keyserver keyserver.ubuntu.com --recv-keys 51716619E084DAB9
sudo apt-get update
sudo apt-get install r-base
\end{minted}
this gives you the baisc features of R, but its true power lies
in the numerous packages. Packages can be installed easily from
within R. Start it up 
\begin{minted}{bash}
R
\end{minted}
and install some packages we will need:
\begin{minted}{R}
install.packages("RPostgreSQL")
install.packages("rgeos")
install.packages("rgdal")
install.packages("sp")
\end{minted}
the first of this command will ask you to select a mirror from
where to dowload the packages. Select one close to your location.

Also python can be used for data analysis some good libraries
to start with are installed using pip:

\begin{minted}{bash}
sudo pip install numpy
sudo pip install scipy
sudo pip install pandas
sudo pip install lxme
sudo pip install sqlalchemy
sudo pip install matplotlib
\end{minted}


\section{PART B. Installation of an RRVS server}

This part describes the installation of a RRVS server hosting
the REM\_RRVS tool and its Google powered pendant GRVS.

\subsection{Before the installation}

We assume a basic experience in linux environments, i.e., familiarity
with a terminal and how to use a package manager furthermore root
privileges are required.
Make sure your repository list and all installed packages 
are up-to-date, run in a terminal:
\begin{minted}{bash}
sudo apt-get update
sudo apt-get upgrade
\end{minted}

\subsection{Tools}
The server can be installed on any Ubuntu 16.04 (a display server, X-server is not necessary.)
The code developed by GFZ is hosted on github, thus the first requirement is git and we will install most python packages using pip.
Install them from the shell using apt-get and upgrade pip using pip:

\begin{minted}{bash}
sudo apt-get install git
sudo apt-get install python-pip
sudo pip install --upgrade pip
\end{minted}

\textit{NOTE: We currently support only Python 2.7}

\subsection{REM Database}
The core of the RRVS server is the REM Database. The database
is a PostgreSQL database with PostGIS extension. Follow the 
instructions in the Routing section in Part A of this installation 
manual to obtain a
PostgreSQL and PostGIS installation on your machine.


%\begin{minted}{bash}
%sudo apt-get install htop
%htop
%\end{minted}
%this shows you all the running processes hit "F4" and type postgres.
%It should show only processes realted to the postgres server.
%In the column "Command" should be one similar to this:
%\begin{minted}{bash}
%/usr/lib/postgresql/9.5/bin/postgres -D /var/lib/postgresql/9.5/main 
%-c config\_file=/etc/postgresql/9.5/main/postgresql.conf
%\end{minted}
%If you cannot see the full command mark the line (arrow key up or down)
% and scroll to the right (arrow key right). The "-D" indicates the 
%location of the datacluster. \textbf{Memorize or write down the location in case the server crashes at some point in the future}


Get the schema for the database from github and clone it to your home directory:
\begin{minted}{bash}
git clone git://github.com/GFZ-Centre-for-Early-Warning/REM\_DBschema ~/REM\_DBschema
\end{minted}

Go to the directory and connect to the database cluster using psql (providing the new password):
\begin{minted}{bash}
cd ~/REM\_DBschema
psql --u postgres
\end{minted}

Then, create a new database called "rem", connect to it, the prompt
should change to "rem=\#" and
start creating the schemas of the REM database from the scripts 
in the cloned directory:
\begin{minted}{bash}
CREATE DATABASE rem;
\c rem
\i create\_templatedb.sql
\end{minted}
Monitor the output printed in your shell there should be no errors
(only short commands) printed while the script is running.

\subsection{RRVS}
The RRVS tool is using the flask webframework for Python and we 
recommend using an Apache2 wsgi server for hosting the application,
thus install apache2 with mod-wsgi on your system, activate wsgi
and create a virtual host for the app:
\begin{minted}{bash}
sudo apt-get install apache2 libapache2-mod-wsgi
sudo a2enmod wsgi
sudo gedit /etc/apache2/sites-available/rrvs.conf
\end{minted}

\subsubsection{Apache2 settings}
In the file create a virtual host by adding the following content:
\begin{minted}{xml}
Listen 80  
<VirtualHost *:80>  
        ServerName localhost  
        WSGIScriptAlias / /var/www/html/rrvs/rrvs.wsgi  
        <Directory /var/www/html/rrvs/webapp>  
            Order allow,deny  
            Allow from all  
        </Directory>
  
        Alias /static /var/www/html/rrvs/webapp/static  
        <Directory /var/www/html/rrvs/webapp/static/>  
            Order allow,deny  
            Allow from all  
        </Directory>  
        ErrorLog ${APACHE\_LOG\_DIR}/error.log  
        LogLevel warn  
        CustomLog ${APACHE\_LOG\_DIR}/access.log combined  
	
	Alias /pano /data/pano                           
	<Directory /data/pano/>       
 	    Options Indexes FollowSymLinks MultiViews
     	    Require all granted
        </Directory>

</VirtualHost>
\end{minted}

The last part sets up a virtual directory from which the panoramic
images will be served we created a directory "/data/pano" but this
can be any location on the system just adjust the path accordingly:
\begin{minted}{bash}
sudo mkdir /data/pano
sudo chmod -R 755 /data
\end{minted}
Make sure the permissions are set to 755 for all directories of 
this path. To secure it from unwanted direct access put a 
file index.html in it with a redirect to the main page of the server:

\begin{minted}{html}
<html class="wf-fontawesome-n4-active wf-active">
<head><meta http-equiv="Content-Type" content="text/html; charset=UTF-8"> 
<title>RRVS</title> 
</head>
<body>
<meta http-equiv="refresh" content="0; url=http://your.server.com" />
</body>
</html>
\end{minted}

After saving the virtual host file activate the site
\begin{minted}{bash}
sudo a2ensite rrvs
\end{minted}

\subsubsection{The webapplication}

The RRVS tool is located in another git repository. Go to
the html directory created by apache on your computer 
and clone the repository there in a directory called rrvs
(note that only root has write permissions there):
\begin{minted}{bash}
cd /var/www/html
sudo git clone git://github.com/GFZ-Centre-for-Early-Warning/REM\_RRVS ./rrvs
cd rrvs
\end{minted}

Install a redis database which is used for server side caching and
the python requirements:
\begin{minted}{bash}
sudo apt-get install redis-server
sudo pip install -r python_requirements.txt
\end{minted}

Now we have to adjust the configuration of the application:
\begin{minted}{bash}
sudo cp rrvs\_config.py.example rrvs\_config.py
sudo gedit rrvs\_config.py
\end{minted}

In the configuration file adjust the following lines:

\begin{minted}{bash}
old: SQLALCHEMY\_DATABASE\_URI = 'postgresql://user:pw@localhost:5432/rem'
new: put the postgres username instead of "user" probably "postgres" 
     and for "pw" the "password" you set before
old: SECRET\_KEY = '42'
new: Set a strong key here, 
     best is to generate a random alphanumeric string
old: DEBUG = True
new: DEBUG = False

\end{minted}

Open the wsgi file
\begin{minted}{bash}
sudo gedit rrvs.wsgi
\end{minted}
 
and verify that "sys.path.append("/var/www/html/rrvs") in line 4 points to the right directory.

Now restart the apache2 server 
\begin{minted}{bash}
sudo service apache2 restart
\end{minted}

\subsubsection{Expanding the REM database for RRVS}
In order to use the perviously created rem database we have to extend
it using an sql script provided with the rrvs repository
go to the scripts directory and start psql:
\begin{minted}{bash}
cd /var/www/html/rrvs/scripts
psql --u postgres
\end{minted}
From within psql, connect to the rem database and run the extension
script:
\begin{minted}{bash}
\c rem;
\i extend\_rem\_db.sql;
\q
\end{minted}
Again monitor the output for error messages.

\subsection{Panoramic images}
In order for the webapplication to find the images put them
in the folder previously defined as the /pano directory in the 
Apache2 server configuration. Our suggestion is to make a 
separate directory for each survey in the pano directory. 

\subsection{Populating the Database for RRVS}
The population of the database is done after a survey is completed.
As the first step we need to define a new survey in the table 
survey.survey of the rem database and populate the image schema. 
We assume that the GFZ Permanent tool has been used to process the 
data collected with the GFZ MoMa system. In this case the survey 
information is created alongside the image resampling and metadata 
creation. We can then simply connect to the database and import
the sql file created by Permanent.

\begin{minted}{bash}
psql --u postgres
\c rem;
\i metadata.sql
\q
\end{minted}
This creates the survey and writes all information about the images 
to the database.

In the next step we populate the asset schema in the database 
with building footprints. Footprints can be obtained from
sources as OpenStreetMap (OSM) directly (via the QGIS plugin)
 or can be digitized from 
satellite/aerial imagery or using the Java application of OSM
(JOSM) with the building tool.


The last missing information before we can generate tasks
for the analysis is to set up users an example can be found
\begin{minted}{bash}
sudo gedit users.sql
\end{minted}

Adjust the file accordingly, i.e., remove or add lines or change the
name of the user in the VALUES: 
\begin{minted}{SQL}
--add test users, roles and tasks
INSERT INTO users.users(id, authenticated, name) VALUES (1, TRUE, 'Test1');
INSERT INTO users.users(id, authenticated, name) VALUES (2, TRUE, 'Test2');

INSERT INTO users.roles(id, name) VALUES (1, 'public');
INSERT INTO users.roles\_users(user\_id, role\_id) VALUES (1, 1);
INSERT INTO users.roles\_users(user\_id, role\_id) VALUES (2, 1);
\end{minted}


After users are set up you should be able to reach the login
page.

If you are prompted with an error check the error file of the apache
server for error messages:

\begin{minted}{bash}
sudo gedit /var/log/apache2/error.log
\end{minted}

\subsection{GRVS}
There is a version of the RRVS which does not require the
collection of omni-directional images but uses Google Streetview
within the app. In order for this to work properly there
are some deviations from the standard RRVS installation.
First of all a Google key is required.
Create a google account and obtain a Google Maps API key
following the instructions here:
\url{https://developers.google.com/maps/documentation/javascript/get-api-key?hl%3C0en}

Once you obtained a key we can start installing the GRVS.
The code is in the same repository as the RRVS.
Get the RRVS code repository and checkout the "grvs" branch 

\begin{minted}{bash}
cd /var/www/html
sudo git clone git://github.com/GFZ-Centre-for-Early-Warning/REM\_RRVS ./grvs
cd grvs
sudo git checkout grvs
\end{minted}

The rest of the installation procedure is similar to the RRVS 
installation. 

\end{document}
